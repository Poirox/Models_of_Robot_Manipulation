%  Robotics Text  by Jacob Rosen and Blake Hannaford
% (c) 2007  Jacob Rosen and Blake Hannaford
%

\chapter{Introduction}
This text has evolved from lecture notes for EE543 at the University of Washington.  It will be evident to the reader that it is still very much a work in progress.  It owes a great deal to the book "Introduction to Robotics, Mechanics and Control," by John Craig.

As you read this book, please report typos or suggestions (by page and line number) to

{\tt https://catalyst.uw.edu/gopost/board/blake/20007/}

\section{Background and Motivational Examples}
 
\section{Definitions}

A robot manipulator is a chain of {\it links} connected by {\it joints}.
\subsection{Degree of Freedom}
A degree of freedom (DOF) is a motion which can be completely described by a scalar and its time derivatives. 

\subsection{Joints}
Joints constrain the relative position of two links in all but one degree of freedom. 

{\it Rotary } joints permit revolute motion about a line in space refered to as the motion axis. 

{\it Prismatic} joints  permit translational motion along a line in space refered to as the motion axis. 

Three key components of joints are actuators, sensors, and bearings.  

Actuators do work on the joint through the application of force or torque.  Most robot manipulators use electric motors as actuators but hydraulic, pneumatic, and other more exotic actuators can also be used. 

Sensors measure the motion of the degree of freedom of the joint and send this information to a control system.

Bearings are devices which restrict motion in all but one degree of freedom. They act as rigid structural elements in the constrained directions and minimize friction in the degree of freedom.

\subsection{End Effector}

The end effector of a robot is a device attached to a link in the chain (typically at the end of a serial chain) designed to accomplish a task.  Typical end effectors include grippers, spray guns, and tools.

\subsection{Joint Space}
\subsection{Task Space}
\subsection{Workspace}


\section{Fundamental Problems in Manipulator Control}

This book will use a large set of mathematical tools to model and analyze robot manipulators.  In learning all of these tools, it will be useful to keep in mind that all of them are used to answer fundamental and straightforward questions which must be answered in order to use robot manpipulators. 

\begin{itemize}
	\item Description of a positioning task

Q:  How do we specify the position and orientation of robot arms and objects?

A: Frames, Homogeneous Transformations, and joint-space teaching.

	\item Forward kinematics

Q: What is the position and orientation (configuration) of a robot end effector if joint positions are known?

A: A linear transformation (4x4 matrix) which is a function of the joint positions (angles, displacements) and specifies end effector configuration in the base frame.  This matrix can also map a point in the end effector frame to its representation in the base frame. 

	\item Inverse kinematics

Q: What joint positions $\theta$, do I need to achieve a given end effector configuration?

A: Form an equation by setting the 4x4 matrix of the foward kinematics problem equal to one specifiing the {\it desired} position and orientation.  Then solve this equation for the joint positions.   Be sure to watch out for existence of solutions and multiple solutions. 


	\item Velocity transformation


Q:  What joint velocities, $\dot{\theta}$, do I need to achieve a given velocity of the end effector (both linear and angular velocity) in all three dimensions of a convenient reference frame?

A: Compute the ``Jacobian Matrix", the matrix derivative, of the forward kinematic equations and invert it.   Watch out for singular matrix in which case the inverse does not exist.
\[
\dot{\theta} = J^{-1} \dot{x}
\]
 
	\item Force transformation

Q:  What joint torques, $\tau$, correspond to a given set of end effector forces and torques, $F$?

A: $\tau = J^T F$
 
	\item Inverse dynamics

Q: For a given motion, $\ddot{\theta}, \dot{\theta}, \theta(t)$, what are the required joint torques?


A:  We can derive a dynamic equation
\[
\tau(t) = M(\theta)\ddot{\theta} + C(\theta,\dot{\theta}) + g(\theta)
\]
by using the Recursive Newton-Euler or Lagrangian methods. 

	\item Position Control

Q: How can we command joint torques to achieve a desired end effector trajectory?

A: PID control, Feed-forward dynamic control (the method of computed torques).

	\item Force Control

Q: How can we command joint torques to achieve a set of forces and torques at the end effector?

A:  Open loop we can use $\tau = J^T F$. However because of friction and other losses in many typical manipulators we must use force sensing at the end effector and closed loop force control.


	\item Trajectory Generation

Q: If we know a trajectory we want to follow, how do we generate a time-function which is a quick but smooth and attainable trajectory between two configurations ``A" and ``B"?

A:  Methods include polynomial splines and trapezoidal velocity profiles.

	\item Motion Planning

Q:  How do we decide the path to take between two configurations ``A" and ``B"?

A:  Without considering obstacles, the most common approach is straight line interpolation.  When obstacles and joint limits must be considered this becomes the AI motion planning problem.

	\item Sensor Based Control

Q: How can I make use of non-joint sensor information (e.g. vision, proximity, etc.) to control robot motion (for example, to follow 1cm above an unknown surface).

A: Robot vision, vision servoing, sensor data fusion, world modeling, operational space control.

	\item Telemanipulation 

Q: How can a human operator control a remote manipulator in a ``natural" and productive way to achieve a task goal?

A: Generalized bi-lateral teleoperation.

\end{itemize}

\section{Summary of Notation}
% Summary of Notation for Chapter  01

This explains the notation of chapter 1.
