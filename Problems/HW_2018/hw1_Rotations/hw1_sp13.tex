%% Default Latex document template
%%
%%  blake@rcs.ee.washington.edu

\documentclass{article}

% Uncomment for bibliog.
%\bibliographystyle{unsrt}

\usepackage{graphicx}
\usepackage{amsmath}

%\usepackage{lineno}
%\linenumbers
%%%%%%%%%%%%%%%%%%%%%%%%%%%%%%%%%%%%%%%%5
%
%  Set Up Margins

%%%%%%%%%%%%%%%%%%%%%%%%%%%%%%%%%%%%%%%%%%%%%%%%%
% include file for:
%      Critical Page setup dimensions
%            DO NOT MODIFY
%       (for help see "Latex Line by Line" p 260)
%
\setlength\oddsidemargin{0in}
\setlength\evensidemargin{0in}

\usepackage[left=0.98in, right=0.98in, top=1.0in, bottom=1.0in]{geometry}

% %Top Margin and header
% \setlength\voffset{-0.94in}
% \setlength\topmargin{0.25in}
% \setlength\headheight{0.25in}
% %\setlength\headwidth{6.5in}
% \setlength\headsep{0.25in}
% %Body
% \setlength\textwidth{6.5in}
% \setlength\textheight{9.50in}
% %Footer
% %\setlength\footheight{0.5in}
% \setlength\footskip{0.3750in}
% Line spacing for 6 lines per inch
\linespread{0.894}  % 1.0 = single    1.6 = double
%
%          END of Critical Page Setup Dimensions
%%%%%%%%%%%%%%%%%%%%%%%%%%%%%%%%%%%%%%%%%%%%%%%%%%%

%%%%%%%%%%%%%%%%%%%%%%%%%%%%%%%%%%%%%%%%%%%%%%%%%%%
%
% Useful style and math macros
%


\newcommand\Dfrac[2]{\frac{\displaystyle #1}{\displaystyle #2}}
\newcommand\beq{\begin{equation}}
\newcommand\eeq{\end{equation}}

\newcommand\bmat{\begin{bmatrix}}
\newcommand\emat{\end{bmatrix}}

\newenvironment{solution}
{\vspace{0.125in} {\bf SOLUTION:} \\ }
{\vspace{0.25in}}





%%%%%%%%%%%%%%%%%%%%%%%%%%%%%%%%%%%%%%%%%%%%%%%%%
%
%         Page format Mods HERE
%
%Mod's to page size for this document
\addtolength\textwidth{0cm}
\addtolength\oddsidemargin{0cm}
\addtolength\headsep{0cm}
\addtolength\textheight{0cm}
%\linespread{0.894}   % 0.894 = 6 lines per inch, 1 = "single",  1.6 = "double"

%\lhead{LEFT HEADER}
%\chead{CENTER HEADER}
%\rhead{RIGHT HEADER}
%\lfoot{Hannaford, U. of Washington}
%\rfoot{\today}
%\cfoot{\thepage}
\begin{document}

\title{EE546 HW 1 \\ University of Washington}
\setcounter{section}{1}


\maketitle

Computer software may be used for any problem.   If computer software is used, please state which package you have used and give a concise script in addition to the result.

\subsection{}          % 1.1
A point in 3-Space is located at
\[
^0P = \bmat 5 \\-7\\3 \emat
\]
in Frame 0 and
\[
{^1P} = \bmat 16\\0\\3\emat \qquad {^3P} = \bmat 1.7\\-6\\-4\emat
\]
Finally, we are given that
\[
{^0_1R} = \bmat .961 & -.276&0\\.234 & .815 &-.530\\.146 & .509 & .848\emat
\]

\subsubsection{}
Find the origin of Frame 1 expressed in Frame 0.  Draw a diagram of the frames (does not have to be to scale) and give the numerical value of $O_1$.

\subsubsection{}
If
\[
{^1_2R} =  I
\]
Find $^1O_2$.



\subsection{}
Give the rotation matrix corresponding to a rotation about $X$ by $\theta$ followed by a rotation about the new $Z$ axis by $\alpha$.




\subsection{}
Frames $A$ and $B$ start out superimposed.  Then
\begin{enumerate}
  \item $B$ is rotated by $\theta_1$ about $X_A$.
  \item $B$ is rotated by $\theta_2$ about $Y_A$.
  \item $B$ is rotated by $\theta_3$ about $X_B$.
\end{enumerate}
Find ${^A_BR}$ after these rotations.  Multiply out the answer in symbolic form.


\subsection{}
Using the result of Section 2.6 (see notes), show that if $K=\bmat 0 & 1 & 0 \emat ^T$ then
\[
Rot(K,\theta) = rot(\hat{y}, \theta)
\]


\subsection{}

The following rotations are performed

\begin{enumerate}
  \item A roll-pitch-yaw rotation in which

   Roll = $16^\circ$, Pitch = $5^\circ$, Yaw = $4^\circ$

  \item A rotation about the current $Y$ axis by $45^\circ$.
  \item A rotation about the original $X$ (Roll) axis by $20^\circ$.
\end{enumerate}

Find the rotation matrix which represents the final orientation.   Multiply out your answer in numerical form.

\end{document}

