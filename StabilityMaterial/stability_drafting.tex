%% Default Latex document template
%%
%%  blake@rcs.ee.washington.edu

\documentclass[letterpaper]{article}

% Uncomment for bibliog.
%\bibliographystyle{unsrt}

\usepackage{graphicx}
\usepackage{lineno}
\usepackage{amsmath}
%\usepackage{fancyhdr}

%%%%%%%%%%%%%%%%%%%%%%%%%%%%%%%%%%%%%%%%5
%
%  Set Up Margins

%%%%%%%%%%%%%%%%%%%%%%%%%%%%%%%%%%%%%%%%%%%%%%%%%
% include file for:
%      Critical Page setup dimensions
%            DO NOT MODIFY
%       (for help see "Latex Line by Line" p 260)
%
\setlength\oddsidemargin{0in}
\setlength\evensidemargin{0in}

\usepackage[left=0.98in, right=0.98in, top=1.0in, bottom=1.0in]{geometry}

% %Top Margin and header
% \setlength\voffset{-0.94in}
% \setlength\topmargin{0.25in}
% \setlength\headheight{0.25in}
% %\setlength\headwidth{6.5in}
% \setlength\headsep{0.25in}
% %Body
% \setlength\textwidth{6.5in}
% \setlength\textheight{9.50in}
% %Footer
% %\setlength\footheight{0.5in}
% \setlength\footskip{0.3750in}
% Line spacing for 6 lines per inch
\linespread{0.894}  % 1.0 = single    1.6 = double
%
%          END of Critical Page Setup Dimensions
%%%%%%%%%%%%%%%%%%%%%%%%%%%%%%%%%%%%%%%%%%%%%%%%%%%

%%%%%%%%%%%%%%%%%%%%%%%%%%%%%%%%%%%%%%%%%%%%%%%%%%%
%
% Useful style and math macros
%


\newcommand\Dfrac[2]{\frac{\displaystyle #1}{\displaystyle #2}}
\newcommand\beq{\begin{equation}}
\newcommand\eeq{\end{equation}}

\newcommand\bmat{\begin{bmatrix}}
\newcommand\emat{\end{bmatrix}}

\newenvironment{solution}
{\vspace{0.125in} {\bf SOLUTION:} \\ }
{\vspace{0.25in}}



%
%        Font selection
%
%\renewcommand{\rmdefault}{ptm}             % Times
%\renewcommand{\rmdefault}{phv}             % Helvetica
%\renewcommand{\rmdefault}{pcr}             % Courier
%\renewcommand{\rmdefault}{pbk}             % Bookman
%\renewcommand{\rmdefault}{pag}             % Avant Garde
%\renewcommand{\rmdefault}{ppl}             % Palatino
%\renewcommand{\rmdefault}{pch}             % Charter


%%%%%%%%%%%%%%%%%%%%%%%%%%%%%%%%%%%%%%%%%%%%%%%%%
%
%         Page format Mods HERE
%
%Mod's to page size for this document
\addtolength\textwidth{0cm}
\addtolength\oddsidemargin{0cm}
\addtolength\headsep{0cm}
\addtolength\textheight{0cm}
%\linespread{0.894}   % 0.894 = 6 lines per inch, 1 = "single",  1.6 = "double"

% header options for fancyhdr

%\pagestyle{fancy}
%\lhead{LEFT HEADER}
%\chead{CENTER HEADER}
%\rhead{RIGHT HEADER}
%\lfoot{Hannaford, U. of Washington}
%\rfoot{\today}
%\cfoot{\thepage}



% Make table rows deeper
%\renewcommand\arraystretch{2.0}% Vertical Row size, 1.0 is for standard spacing)

\begin{document}
\setpagewiselinenumbers        %  Line numbers for edits to drafts.
\modulolinenumbers[1]          %  number every N lines
 
\section{Ultra-simplified linear block diagram}
\beq
\mathrm{H.O.} = \mathrm{Mechanical~Admittance} = \frac{\dot{x}_h(s)}{F_h{s}} = A_{HO}(s)
\eeq
\beq
\mathrm{Env} = \mathrm{Mechanical~Impedance} = \frac{F_E(s)}{\dot{x}_E(s)} = Z_{Env}(s)
\eeq
Loop Gain:
\beq
G(s) = A_{HO}(s)Z_{Env}(s)H_1(s)H_2(s)
\eeq

Characteristic equation of Closed Loop system:
\beq
1+G(s)  = 0
\eeq

Nyquist criteria:
\beq
|G(s)| \geq 1  \quad \mathrm{and} \quad \angle{G(s)} \geq 180^\circ
\eeq

Assume: 
\[
|H_1(s)|  = |H_2(s)| = 1, \quad \angle{H_1(s)H_2(s)} \approx 180^\circ
\]
Then 
\beq
|G(s)| = |Z_{Env} A_{HO}| = \left |\frac{Z_{Env}}{Z_{HO}} \right |,\quad \angle{G(s)} \approx 180^\circ
\eeq
Stability is determined by impedance ratio:
\beq
\left |\frac{Z_{Env}}{Z_{HO}} \right | \stackrel{?}{>} 1
\eeq

Delay:
\beq
\angle{Z_{Env}} - \angle{Z_{Env}} + \angle{e^{-sT}} \geq 180^\circ
\eeq

What about phase shift assumption?  Think of 
\begin{itemize}
 \item Sampling
 \item Computation delay
 \item Communication delay
 \item Phase Lag in mechanisms, environment
 \item Cognitive, biomechanical and reflex delays in human
\end{itemize}

\section{Stabilization Approaches}
\subsection{Virtual Coupling}
Ed Colgate et al. 
\subsection{Passivity}
A system having one port, with variables $\dot{x}$, and $F$ is passive if
\beq
\int_{-\infty}^{t} F(t)\dot{x}(t)dt \geq 0 \quad \forall t
\eeq
or
\beq
\int_{0}^{t} F(t)\dot{x}(t)dt + E(0) \geq 0 \quad \forall t>0
\eeq
Where
\beq
E(0) = \int_{-\infty}^{0} F(t)\dot{x}(t)dt
\eeq
For a two port system, the equation becomes


\beq
\int_{0}^{t} \left (F_1(t)\dot{x}_1(t)dt +F_2(t)\dot{x}_2(t)dt\right) + E(0) \geq 0 \quad \forall t>0
\eeq

Llewellyn's Absolute stability:

If a two port system has matrix description 
\[
\left [
\begin{array}{cc}
    p_{11} & p_{12} \\
    p_{21} & p_{22}
\end{array}
\right ]
\]
Then it is stable for any pair of 1-port terminations iff
\beq\label{LlewellynCrit}
\begin{aligned}
\Re(p_{11}) &\geq 0 \\
\Re(p_{22}) &\geq 0 \\
2\Re(p_{11})\Re(p_{22}) - |p_{12}p_{21}| - \Re(p_{12}p_{21}) &\geq 0 
\end{aligned}
\eeq

\section{Passivity Observer / Controller }

\subsection{Passivity Observer}
Consider a discrete time one-port system.  We can refer to its port variables, 
$F(k)$ and $\dot{x}(k)$ as conjugate variables (their product is power).
The passivity definition of the one port is
\beq\label{POineq}
\sum_{k=1}^{n} F(k)\dot{x}(k) + E(0) \geq 0 \quad \forall n>0
\eeq

If the two energy variables $F(k), \dot{x}(k)$ are monitored by sensors and fed into a computer system, we can observe this energy as
\beq
E_{obsv}(n) = \sum_{k=1}^{n} F(k)\dot{x}(k) + E(0) 
\eeq
Software can then determine any instant of time $n$ at which the system violates 
(\ref{POineq}).  This ``passivity observer" is shown in the figure 

%    01104

A similar observer can be developed for multiple ports.  For $m$ ports, 

\beq
E_{obsv} =  \sum_{p=1}^m \sum_{k=1}^{n} F_p(k)\dot{x}_p(k) + E(0)  
\eeq

This is illustrated in figure

%  01105

\subsection{Passivity Controller}  Going back to the one-port, since such systems often have control over one of their conjugate variables (i.e. an output), why not modify 
the output in real time to constrain 
\[
E_{obsv} > 0
\]

To do this we introduce a ``Passivity Controller" (PC) into the system which modifies the computed output.  There are two forms we can use, the serial PC and the parallel PC. 

The Series PC can be described by the following algorithm which computes a modified output force 
based on desired/computed force and current velocity:
\begin{enumerate}
    \item $\dot{x}(k)$ is an input measured velocity
    \item $F_c(k)$ is the output computed by the system block
    \item $E_{obsv}(n) = E_{obsv}(n-1)+[F_c(n)\dot{x}(n)+\alpha(n-1)\dot{x}(n-1)^2]\Delta T$
    \item  
    $\alpha(n) = \begin{cases} 
       -E_{obsv}(n)/(\Delta T\dot{x}(n)^2 & E_{obsv}(n) < 0 \\
       0                                 & E_{obsv}(n) \geq 0
       \end{cases}
       $
    \item $F_{out}(n) = F_c(n)+\alpha(n)\dot{x}_n$
\end{enumerate}


- - - - - - - - - - - - - - - - - - - - - - - - - - - - 

\section{2017 ACC Workshop}
\subsection{Hashtrudi-Zaad: (2001)}

Position-position

2-port can be active but absolute stability 

Haykin '70:    Stability-Activity Diagram  (Graphical version of the inequality (\ref{LlewellynCrit})

Limitations:   Only passive environments.  

\subsubsection{``Circle method"}  Edwards '92, Haddadi et al 2010, Haddadi et al. 2015

Scattering Representation / Scattering Matrix  /  Reflection Coefficients


Passivity:  unit disk in scattering domain

Environment stability region / all active or passive env's for which combined system is passive. 

ESR a function of frequency:  a circle which is function of scattering params.

(overlap of circles at each frequency )  Does ESR cover unit passivity disk?

Now move from scattering domain to impedance params. 

Application: Transparency Optimized 4-Channel Controller  (ToC)

Absolute stability - requires acceleration estimate  

\subsubsection{Adding Human Operator Damping}
Haddadi et al. 2015.

With acceleration:
As operator adds more damping, ESR region grows. 

Without acceleration:
There is no damping for which all passive envs are stable. 

\subsection{Absolute Stability of multilateral systems and multi-DOF systems}
M. Tavakholi

Application: tele-rehab

Apply Llewellyn to 3-port terminated by 3 $Z_i$s.

Definition:  Youla Passive N-Port IRE 1959.

Use Youla's lemmas: reciprocal N-port matrices. 

Raisbeck's criterion  (more conservative.)

Derivation of Counterpart to Llewellyn for 3-ports and counterpart to Raisbeck's criterion.

N-port, M-DOF system (linearized by Feedback Linearization)

``Complementary" DOF (IET Control Theory \& Applications 2014).  Contact tasks constraining some DOFs.

Peg-in-hole example: i.e Master one for free motion, master two for peg insertion


\subsection{Energy Aware Robotics} Stefano Stramigioli

"casmir function for control" IFAC 1998

Port-based thinking

world is non-linear

Proper geometry makes multi-dimensional systems as simple as 1-D

A model is not reality:  can't base stability on "environment model"

``Passivity" is not conservative as often thought -- think instead about Energy Awareness.

Electrical Networks give misleading topology - doesn't represent energy flow. 

Stability is not difficult if we respect physics. / "listening to the math and physics" can avoid problems.

Contrast non-interactive control (e.g. sattelite dish) vs interactive control (robot holding object). Not the same as Robust Control.

example: bi-ped control by "system composition" e.g. bond-graphs

Examples: DLR grasping 

\subsection{Stability using Multiplier THeory and IQC}  J. Carrasco, H. Tugal, Univ of Manchester

Assumptions: Human is Passive. Environment = Force = F(V)

"bi-lateral teleoperation IQC formulation" Ilham Polat 2012

Assume environment is monotone static nonlinear operator (Passive plus more)

convex search using "IQC" for controller. 

``Integral Quadratic Constraint" (IQC):  Megretski 1997 IQC theorem  (cf Multiplier theory, Desoer and Vidyasagar 1975)

Zames-Falb Muiltpliers for absolute stability, Carrasco et all EJC 2017.

\subsection{Passivity and Life Hearafter} Gunter Niemeyer, Disney Research

Goal checkup:  Transparency ? or User Projection ?     Leverage wider robotics community.

Event based haptics.

Allure of Passivity. ``all natural"

lack of model is strength but also weakness of passivity.


\begin{itemize}  \item HF ($>$30Hz)    open loop
\item MF (3Hz)  Passivity
\item LF ($<$1Hz) Active may be OK
\end{itemize}

Experiment: metal-to-metal contact with 4 sec time delay

Intent is encoded in human's grasp impedance.  

GN idea/video: Squeeze sensor on gripper driving slave stiffness.

\subsection{} (Ilia Polushin, Western University, Canada)

Induced Master Motion - (Kukhenbecker \& Niemeyer, 2006)

Decouple the reflected force into two parts: "Interaction component" (coupled to HO)  \& "motion inducing component" (causes feedback loop to displacement and instability.

Seek a smart aglorithm which would attenuate only the motion inducing component.

Idea:  project feedback force onto HO force.   Dot product is the interaction component. 


\end{document}

