% Problem Statement and Learning Objectives for Chapter 04
\paragraph{Problem Statement}
  
The inverse kinematics problem is to find the joint values for a given end effector configuration.  The solution to this problem is needed for most practical applications of serial robot arms. 

  
\paragraph{Learning Objectives}
Upon completing this Chapter, the reader should

\begin{itemize}
	\item Be able to plot the workspace of a serial manipulator with or without joint limits.
	\item Be able to solve for the joint angles of a planar robot given the end effector $x,y$ position and orientation.
	\item Be able to solve for the joint angles of a spatial 3 DOF robot given the $x,y,z$ position of the end effector.
	\item Be ready to tackle full scale inverse kinematics problems if sufficient time is available. 
\end{itemize}